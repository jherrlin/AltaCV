%%%%%%%%%%%%%%%%%
% This is an sample CV template created using altacv.cls
% (v1.6.4, 13 Nov 2021) written by LianTze Lim (liantze@gmail.com). Now compiles with pdfLaTeX, XeLaTeX and LuaLaTeX.
%
%% It may be distributed and/or modified under the
%% conditions of the LaTeX Project Public License, either version 1.3
%% of this license or (at your option) any later version.
%% The latest version of this license is in
%%    http://www.latex-project.org/lppl.txt
%% and version 1.3 or later is part of all distributions of LaTeX
%% version 2003/12/01 or later.
%%%%%%%%%%%%%%%%

%% Use the "normalphoto" option if you want a normal photo instead of cropped to a circle
% \documentclass[10pt,a4paper,normalphoto]{altacv}

\documentclass[10pt,a4paper,ragged2e,withhyper]{altacv}
%% AltaCV uses the fontawesome5 and packages.
%% See http://texdoc.net/pkg/fontawesome5 for full list of symbols.

% Change the page layout if you need to
\geometry{left=1.25cm,right=1.25cm,top=1.5cm,bottom=1.5cm,columnsep=1.2cm}

% The paracol package lets you typeset columns of text in parallel
\usepackage{paracol}

% Change the font if you want to, depending on whether
% you're using pdflatex or xelatex/lualatex
\ifxetexorluatex
  % If using xelatex or lualatex:
  \setmainfont{Roboto Slab}
  \setsansfont{Lato}
  \renewcommand{\familydefault}{\sfdefault}
\else
  % If using pdflatex:
  \usepackage[rm]{roboto}
  \usepackage[defaultsans]{lato}
  % \usepackage{sourcesanspro}
  \renewcommand{\familydefault}{\sfdefault}
\fi

% Change the colours if you want to
\definecolor{SlateGrey}{HTML}{2E2E2E}
\definecolor{LightGrey}{HTML}{666666}
\definecolor{DarkPastelRed}{HTML}{450808}
\definecolor{PastelRed}{HTML}{8F0D0D}
\definecolor{GoldenEarth}{HTML}{E7D192}
\colorlet{name}{black}
\colorlet{tagline}{PastelRed}
\colorlet{heading}{DarkPastelRed}
\colorlet{headingrule}{GoldenEarth}
\colorlet{subheading}{PastelRed}
\colorlet{accent}{PastelRed}
\colorlet{emphasis}{SlateGrey}
\colorlet{body}{LightGrey}

% Change some fonts, if necessary
\renewcommand{\namefont}{\Huge\rmfamily\bfseries}
\renewcommand{\personalinfofont}{\footnotesize}
\renewcommand{\cvsectionfont}{\LARGE\rmfamily\bfseries}
\renewcommand{\cvsubsectionfont}{\large\bfseries}


% Change the bullets for itemize and rating marker
% for \cvskill if you want to
\renewcommand{\itemmarker}{{\small\textbullet}}
\renewcommand{\ratingmarker}{\faCircle}

%% Use (and optionally edit if necessary) this .cfg if you
%% want to use an author-year reference style like APA(6)
%% for your publication list
\input{pubs-authoryear.cfg}

%% Use (and optionally edit if necessary) this .cfg if you
%% want an originally numerical reference style like IEEE
%% for your publication list
% \input{pubs-num.cfg}

%% sample.bib contains your publications
\addbibresource{sample.bib}

\begin{document}
\name{John Herrlin}
\tagline{Software Developer}
%% You can add multiple photos on the left or right
%\photoR{2.8cm}{Globe_High}
\photoR{2.8cm}{john2}
% \photoL{2.5cm}{Yacht_High,Suitcase_High}

\personalinfo{%
  % Not all of these are required!
  \email{jherrlin@gmail.com}
  \phone{0046 73 73 33 787}
  \location{Växjö / Malmö, Sweden}
  \homepage{jherrlin.github.io/}
  \twitter{@jherrlin}
  \linkedin{john-herrlin-63458280}
  \github{jherrlin}
  %% \orcid{0000-0000-0000-0000}
  %% You can add your own arbitrary detail with
  %% \printinfo{symbol}{detail}[optional hyperlink prefix]
  % \printinfo{\faPaw}{Hey ho!}[https://example.com/]
  %% Or you can declare your own field with
  %% \NewInfoFiled{fieldname}{symbol}[optional hyperlink prefix] and use it:
  % \NewInfoField{gitlab}{\faGitlab}[https://gitlab.com/]
  % \gitlab{your_id}
  %%
  %% For services and platforms like Mastodon where there isn't a
  %% straightforward relation between the user ID/nickname and the hyperlink,
  %% you can use \printinfo directly e.g.
  % \printinfo{\faMastodon}{@username@instace}[https://instance.url/@username]
  %% But if you absolutely want to create new dedicated info fields for
  %% such platforms, then use \NewInfoField* with a star:
  % \NewInfoField*{mastodon}{\faMastodon}
  %% then you can use \mastodon, with TWO arguments where the 2nd argument is
  %% the full hyperlink.
  % \mastodon{@username@instance}{https://instance.url/@username}
}

\makecvheader
%% Depending on your tastes, you may want to make fonts of itemize environments slightly smaller
% \AtBeginEnvironment{itemize}{\small}

%% Set the left/right column width ratio to 6:4.
\columnratio{0.6}

% Start a 2-column paracol. Both the left and right columns will automatically
% break across pages if things get too long.
\begin{paracol}{2}
\cvsection{Experience}

\cvevent{ Software Developer }{Griffin}{Feb 2024 -- Mar 2025}{London (remote), Sweden}

Built a case management system for the bank.

\cvtag{ClojureScript}
\cvtag{Clojure}
\cvtag{Re-frame}
\cvtag{React}
\cvtag{Property based testing}

\divider


\cvevent{ Software Developer }{LeoVegas Group}{Jul 2022 -- Feb 2024}{Växjö (hybrid), Sweden}

Part of the payment team with a focus on card payments.

\cvtag{Kotlin}
\cvtag{Clojure}
\cvtag{Microservices}
\cvtag{Relational databases}

\divider

\cvevent{ System Architect / Software Developer }{Fortnox AB}{Aug 2021 -- Jul 2022}{Växjö (hybrid), Sweden}

Part of two teams. Architect team with a focus on technical progression and
maintenance in the organization. Other team has a focus on internal developer
tools.

\cvtag{RxJava}
\cvtag{React}
\cvtag{TypeScript}
\cvtag{Microservices}
\cvtag{Relational databases}

\divider

\cvevent{Software Developer / Scrum Master}{Fortnox AB}{Aug 2020 -- Aug 2021}{Växjö, Sweden}

Web based system for accounting.

\cvtag{RxJava}
\cvtag{React}
\cvtag{React Native}
\cvtag{TypeScript}
\cvtag{Microservices}
%% \cvtag{PHP}
%% \cvtag{BackboneJS}

\divider

\cvevent{Software Developer}{Herrlin Software}{May 2020 -- Ongoing}{Växjö, Sweden}

Sole proprietorship

\cvtag{Clojure/Script}
\cvtag{Re-frame}
\cvtag{Reitit}
\cvtag{Datomic}

\divider

\cvevent{Software Developer}{KP System AB}{May 2018 -- Aug 2020}{Växjö, Sweden}

Web based system for planning excavation work. \\

\cvtag{Clojure/Script} \cvtag{React} \cvtag{Om}

\divider

\cvevent{Software Developer Consultant}{Sigma IT Consulting}{Jun 2017 -- May 2018}{Växjö, Sweden}

Web based system for managing and ordering spare parts.

\cvtag{Ruby on Rails} \cvtag{CoffeeScript}

%% \divider
%% \cvevent{Software Developer (summer position)}{TECNAU}{Jun 2016 -- Aug 2016}{Ljungby, Sweden}
%% Developing a desktop application for monitoring a printer.
%% \cvtag{NI DAQ} \cvtag{Python} \cvtag{PyQt5}

%% \divider

%% \cvevent{Software Developer / Project manager}{APEA Mobile Security Solutions AB}{Apr 2014 -- Jan 2016}{Växjö, Sweden}

%% Early days startup within the fire alarm sector.

%% \cvtag{Java} \cvtag{Android} \cvtag{Python} \cvtag{Django}

%% \newpage

\cvsection{Voluntary Work}

\cvevent{Organizer}{VXODEV book circle}{Sep 2021 -- Ongoing}{Växjö, Sweden}

Book circle with focus on computer science books.

\divider

\cvevent{Founder}{Kodkollektivet}{Jun 2015 -- Ongoing}{Linnaeus University}

Student association with a IT focus. Social events with companies in a combination
with hackathons, workshops and meetups.

\begin{itemize}
\item President until Nov 2017
\item From Nov 2017 advisor to the board
\end{itemize}

\divider

\cvevent{Organizer}{Wexio Lambda Session / Växjö Functional Programming}{Oct 2020 -- Ongoing}{Växjö, Sweden}

Meetup with a focus on functional programming.

https://www.meetup.com/Vaxjo-Functional-Programming/

\medskip

%% \cvsection{A Day of My Life}

%% % Adapted from @Jake's answer from http://tex.stackexchange.com/a/82729/226
%% % \wheelchart{outer radius}{inner radius}{
%% % comma-separated list of value/text width/color/detail}
%% \wheelchart{1.5cm}{0.5cm}{%
%%   8/8em/accent!10/Work,
%%   6/8em/accent!30/Family,
%%   1/8em/accent!50/Reading,
%%   1/8em/accent!70/Guitar,
%%   1/8em/accent!90/Training
%%   %% 5/6em/accent!20/Playing guitar
%% }

% use ONLY \newpage if you want to force a page break for
% ONLY the current column
%% \newpage

%% \cvsection{Publications}

%% \nocite{*}

%% \printbibliography[heading=pubtype,title={\printinfo{\faBook}{Books}},type=book]

%% \divider

%% \printbibliography[heading=pubtype,title={\printinfo{\faFile*[regular]}{Journal Articles}},type=article]

%% \divider

%% \printbibliography[heading=pubtype,title={\printinfo{\faUsers}{Conference Proceedings}},type=inproceedings]

%% Switch to the right column. This will now automatically move to the second
%% page if the content is too long.
\switchcolumn

\cvsection{My Life Philosophy}

\begin{quote}
``Solving problems together makes everything better.''
\end{quote}

\cvsection{Most Proud of}

\cvachievement{\faHeartbeat}{Emacs / Org mode contributions}{Contributions to free software project}

\divider

\cvachievement{\faHeartbeat}{VXODEVs book circle}{Book circle for CS books}

\divider

\cvachievement{\faHeartbeat}{Student union founder}{Kodkollektivet, student association}

\cvsection{Skills}

\cvtag{Clojure/Script}
\cvtag{Emacs}
\cvtag{Linux}

\divider\smallskip

\cvtag{Git}
\cvtag{Kotlin}
\cvtag{Java}
\cvtag{Functional programming}

\cvsection{Languages}

\cvskill{Swedish}{5}
\divider

\cvskill{English}{4}
\divider

\cvskill{Norwegian}{2.5}
\divider

\cvskill{Danish}{2}


%% Yeah I didn't spend too much time making all the
%% spacing consistent... sorry. Use \smallskip, \medskip,
%% \bigskip, \vspace etc to make adjustments.
\medskip

\cvsection{Education}

\cvevent{Bachelor studies, Computer Science}{Linnaeus University}{Sept 2014 -- June 2017}{}

\cvsection{Certifications}

\cvevent{Professional Scrum Master I (PSM I)}{scrum.org}{Apr 2021}{}


% \divider

%% \cvsection{Referees}

%% % \cvref{name}{email}{mailing address}
%% \cvref{Prof.\ Alpha Beta}{Institute}{a.beta@university.edu}
%% {Address Line 1\\Address line 2}

%% \divider

%% \cvref{Prof.\ Gamma Delta}{Institute}{g.delta@university.edu}
%% {Address Line 1\\Address line 2}


\end{paracol}

%% \newpage

%% \cvsection{About me}

%% Born and raised in Växjö Sweden. After high school I went to Norway to work. In
%% 2012 I came back to Sweden to start living in Malmö and work at a Apple reseller
%% in Copenhagen. After some time in Malmö I decided to start my university studies
%% in my hometown and moved back to Växjö. After some back and fourth I decided to
%% start with a Bachelor degree in Computer Science.

%% While being a student I lacked a social and fun place for IT students to hang
%% out and share knowledge so me and a couple of friends started a student union
%% called Kodkollektivet. Kodkollektivet was a success and we did many great event.

%% I have a passion in software development and people. I feel the most thrill when
%% people are solving problems together and creates really good solutions.

%% Spending much of my time in Emacs and cheers the free software movement.



\end{document}
